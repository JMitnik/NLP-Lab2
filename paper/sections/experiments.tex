\section{Experiments}
\label{sec: experiments}
The main task of this research is to do the sentiment prediction on sentence
level given a train set. The dataset used is the Stanford Sentiment
Treebank\cite{socher2013recursive}, which provides a collection of sentences,
coupled tree structure and fine-grained sentiment scores for subtrees at each
node.
% This dataset consists of 10.662 excerpts
% from longer-form movie reviews. Over all the sentences, a total of 215.154
% annotated phrases can be found. 
%  Each of these reviews have an associated
% sentiment score from the originating review, and each of these phrases have an
% associated sentiment score through human annotation. As such, this dataset can
% capture a semblance of fine-grained sentiment per-phrase and per-sentence.

% To Jonathan: the supervision of this dataset could be illustrated more
% accurately, and also we need to shrink down text, so I replace them with a
% gist.

Each model has been trained three times under different random seeds. These
seeds are 7, 42 and 1984. Table \ref{tab:parameters} represents the number of
times each model is trained, as well as the learning rate of the optimizer and
[number of parameters]. Most of the model will have its embedding projector
initialized with the pre-trained GloVe\cite{pennington2014glove} word embedding
of size 300, without finetune. For other models we use Xavier initialization by
default. Adam optimizer\cite{DBLP:journals/corr/KingmaB14} is used, with
momentum parameters being $\beta_{1}=0.9$ and $\beta_{2}=0.999$ and $eps=1e-08$.
%  Adam is a variant on Stochastic Gradient Descent, which uses an exponential moving
% average of the gradient and squared gradient to control decay of each parameter.
% The loss function is the Cross Entropy Loss for each model.

% During the training, a model is given a sentence with the corresponding label
% from the corpus. This label will be the truth value and used to supervise the
% model during training.  
In almost all the models we use the label at the root node as the supervision to
entire sentence. Only in Subtree-LSTM we supervise every subtree with the root
node of itself, by augmenting the original dataset with all extracted subtrees.
In the test phase, the model performance will be evaluated using [accuracy],
along with a significance test to determine whether the difference is
statistically significant.

% For research question 3, an additional separate trial will be run to train
% Tree-LSTMs.
%  The Tree-LSTM models are supervised per node, instead of on the root
% node. [This is done by extracting all subtrees using an external algorithm
% written for this purpose.]
% TODO: Gongze, do we reference this library, or deny its existence and just 
% assume they believe us?

For research question 4, we put the test-set into a number of bins of size 5,
according to their length of range 5, e.g. sentences of length 1-5, 5-10, and so
on. We report the accuracy of each model on every bin and plot them in one figure.

% TODO: Anymore experiments can go here.