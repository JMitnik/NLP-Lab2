\section{Background}
\label{sec: background}

\subsubsection{Word Embeddings}
Word embedding is a fundamental technique to represent words in an efficient
way. The basic idea is to cast all words in the vocabulary to a low dimensional
real space in a fashion that the more close two words semantically, the more
close they are in terms of Euclidean distance in the embedding space. There are
several embedding types such as word2vec\cite{mikolov2013distributed} and
GloVe\cite{pennington2014glove}, all based on estimating the neighbor words
distribution. We use GloVe in this paper.
 

\subsubsection{BOW, CBOW and Deep CBOW}
The bag of words (BOW) model uses maximum likelihood to predict the most likely
sentiment class. This language model has no representation of word-order and
thus has been considered a simple but not sufficient representation to
effectively capture word-order. The Continuous bag of word (CBOW)
model\cite{DBLP:journals/corr/abs-1301-3781} is an extension of the BOW by
changing the dimensionality of each of the vectors it sums to an arbitrary size
(here chosen to be 300) to house some semantic representation of each word.
Furthermore, to ensure proper calculations, the model learns the corresponding
mapping of these semantic representations to output classes by training a
one-layer perceptron. Furthermore, the Deep CBOW model uses the benfit of a
multi-layer perceptron to encode more information about a word's relationship to
the supervising signal (for instance, sentiment). These models result in
gradients on which an optimizer can perform gradient descent. To derive the
gradient from these perceptrons to perform stochastic gradient descent on, the
Pytorch library can perform automatic differentiation\cite{paszke2017automatic}.

\subsubsection{RNN and LSTM}
When parsing sentences of arbitrary lengths, and attempting some inference based
on these sentences, Recurrent Neural Networks (RNNs) can [interpret] this
sequential data by learning a number of weights matrices. RNNs consist of unit
cells, which receive an input $x_t$, and the 'hidden' state $h_t$ of the
previous unit cell on t-1. This hidden state uses both of these inputs, performs
a linear transformation and a nonlinear transformation such as the tanh
function, and then sends the hidden state to the next cell. Unfortunately, one of
the major difficulties of RNNs is the exponential decay of the gradient vector
over the length of a sentence\cite{bengio1994learning}. This is more commonly
referred to as the vanishing gradient problem.

Long-short term network (LSTM) models \cite{hochreiter1997long} were introduced
in 1997, and added an additional cell state, $c_t$, to the core dynamics of the
RNN. The idea of this cell-state is to counter this major problem of RNN's. An
LSTM unit's cell state serves as long-term memory, and any operation on this
cell state is protected by gates. Each of these gates carefully regulates
interactions with the cell states, such as updating or receiving information.
Via this construction of gate access, the vanishing gradient problem is
minimized.
\subsubsection{Tree LSTM}
While Recurrent Neural Networks and LSTMs process previous unit cells
sequentially, Tree LSTM cells \cite{DBLP:journals/corr/TaiSM15}
\cite{DBLP:journals/corr/ZhuSG15} \cite{DBLP:journals/corr/LeZ15} contain
multiple children cells from which each cell Havcan obtain past knowledge. [Tai et
al] suggest two variants on how these child cells can be implemented: Child-Sum
Tree LSTMs and N-Ary Trees. For a Child-sum LSTM cell, a summation of all hidden
k hidden states is calculated. Each gate vector in this variant is calculated
similarly to regular LSTMs, except the LSTM stores an additional weight matrix
per gate for how to incorporate all the previous solutions, and order of
children is not deemed important. For N-Ary trees, a unit cell additionally
receives a previous memory cell. Instead of summing the previous hidden states
beforehand, this variant does that in each gate with a weight matrix per
assigned child. For instance, calculation of the input gate is altered to the
form $i_j=\sigma(W^i*x_j + \sum\limits_{k=1}^{N}U_k^{i}h_{jl}+b^i)$. Kai et al
state that this allows for more granular calculations, which can result in even
more fine-tuned results. Furthermore, forget gates are calculated per child,
which means that each child has assigned a weight matrix to each other child,
and thus can find interaction via their parent. 