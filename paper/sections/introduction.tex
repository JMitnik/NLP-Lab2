\section{Introduction}
% Pages-size: 1
\label{sec: intro}
Sentiment analysis is a sub-domain within the field of Natural Language
Processing, in which a statistical model infers the direction and magnitude of
emotions from certain corpora. Language representation plays an important role
in the sentiment inference of unseen text. Successful inference depends on
deciding appropriate language models which can model the structural dependencies
efficiently. One such structural dependency is word order. Another complicated
aspect of inference is sufficient context when presented with data to predict or
train on: for instance, if the model has been trained on partial phrases, it
might become easier for the model to make wrong inferences based on non-negated
sentiments. For this reason, it is the goal of this paper to investigate the
role of these dependencies.
% To Jonathan: Is the last obstacle to inference the imbalance between class? If yes, did we do an investigation on this phenomenon in this paper?

This paper will attempt to answer the questions below corresponding to the
[data by examining a number of different research questions.]
% TODO: Reflow the bracketed text
\begin{enumerate}
    \item \textbf{How important is word order for the task of classification of
    sentiment?} Some models, such as the traditional Bag of Words, disregard
    word order in favour of a simpler vector representation of the underlying
    text. Long short-term memory (LSTM), a more recent variant on the Recurrent
    Neural Network, maintains a memory state of recent words encountered to
    incorporate word-order of short and long distances. This paper initially
    believes that models which maintain and understand word-order will perform
    better. 
    % TODO: Do something about the findings
    \item \textbf{Does tree structure improve the accuracy as opposed to
    sequential learning?} Vanilla LSTM as originally introduced, takes in words
    in a sequential manner. One alternative to LSTM is Tree LSTM, which can tree
    structure to obtain phrase representation both globally and locally,
    [enhancing the context available for the model]. The paper seeks to
    investigate whether aggregating context structurally will increase the
    performance of the inference on the test set. The original belief is that
    this will certainly be the case, and as the results suggest, this [seems to
    be the case].
    % TODO: Add cite for sequential introduction TODO: Add cite for tree
    % introduction TODO: Reflow bracketed text. TODO: Is this significant?
    \item \textbf{Performance on supervising sentiment per node?}
    % This paper is seeking to investigate if the Tree LSTM algorithm should be
    % trained on the supervision of the sentiment of individual words, or on the
    % sentiment of the sentences altogether. 
    This paper is seeking to investigate whether adding additional supervision
    on intermediate words or phrases will help the model capture the more
    context-related sentiment. The initial hypothesis of this paper argues that
    the supervision of individual nodes might decrease performance, due to less
    context to account for negating sentiment, for instance.
    % TODO: @Gongze, do you agree with the above initial hypothesis and
    % reasoning? ||| I want to add that the supervision is acted upon not only
    % word level, rather on a phrase level. And also the later result might
    % suggest a neutral effect of adding supervision, that is, no big gap
    % between these two cases. I suspect it because we don't train it with
    % proper learning rate or not until they converge. |||TODO: Incorporate
    % later results.
    \item \textbf{Will the performance vary much on different sentence length across models?}
    % A model's performance might depend on how many words the model needs to keep
    % in store. Decreasing the sentence length might show the difference in model
    % performance, and which model relatively performs better with longer
    % sentences. Initially, this paper argues that Tree LSTM might perform the
    % best, as it has more cells to infer relevant knowledge from than regular
    % LSTM. 
    A model may behave very differently to sentences of different length. On one
    hand, it is because for longer sentences the sentiment semantics is more
    complex and sometimes contradict itself in different parts, on the other
    hand, it is due to the imbalanced distribution of sentences of different
    length in the train set. Therefore it is a crucial problem whether a model
    performs evenly on different length sentences and whether some model
    performs better on capturing sentiments contexts in longer timescale.
    Initially, this paper argues that Tree LSTM might perform the best, as it
    processes the context in a hand-crafted structural way, thus more efficient
    when dealing with long-scale semantics.
    % TODO: @Gongze, do you agree with the initial hypothesis? TODO: Add results
    % Might have improper grammar and usage of word.
    \item \textbf{How does N-ary Tree LSTM compare to the Child-Sum Tree LSTM?}
    The N-ary tree-LSTM differs with Child-sum tree-LSTM in that the latter is
    [agnostic] of the order of the child nodes. On implementation details, the
    N-ary tree-LSTM has different gate parameter for each child, while Child-sum
    use the same parameter for all the gate including forget gate. In most
    cases, N-ary tree-LSTM first project then sum, while the Child-sum tree-LSTM
    do the opposite. Hence they are suitable for different tasks: the N-ary
    tree-LSTM is suitable to those trees where child nodes are ordered, and the
    Child-sum tree-LSTM should work well on the order of child nodes does not
    contribute to the context. In the sentiment classification task, we need
    N-ary tree-LSTM since the word order is important to context extraction.
\end{enumerate}

% TODO: Add citation for Glove
% To Jonathan: is this paragraph necessary there? Because we will touch these in details soon. Maybe we should mention our findings briefly.
In order to address these questions above, the different models mentioned will
be initialized with 300-dimensional GloVe\cite{pennington2014glove} vectors. The
goal is to predict the sentiment of [Stanford Sentiment Treebank data], with the
different settings corresponding to the [aforementioned research questions].
% SST doesn't have a paper to cite, weird.