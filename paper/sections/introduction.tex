\section{Introduction}
% Pages-size: 1
\label{sec: intro}
Sentiment analysis is a sub-domain within the field of Natural Language
Processing, in which an statistical model infers the direction and magnitude of
emotions from certain corpora. Language representation plays an important role
in the sentiment inference of unseen text. Successful inference depends on
deciding appropriate language models which can model the structural dependencies
within the sentences of a language sufficiently. One such structural dependency
is word order. Another complicated aspect of inference is sufficient context
when presented with data to predict or train on: for instance, if the model has
been trained on partial phrases, it might become easier for the model to make
wrong inferences based on non-negated sentiments. For this reason, it is the
goal of this paper to investigate the role of these dependencies.

This paper will attempt to answer some of the questions corresponding to the
[data by examining a number of different research questions.]
% TODO: Reflow the bracketed text
\begin{enumerate}
    \item \textbf{How important is word order for the task of classification of
    sentiment?} Some models, such as the traditional Bag of Words, disregard
    word order in favour of a simpler vector representation of the underlying
    text. Long short-term memory (LSTM), a more recent variant on the Recurrent
    Neural Network, maintains a memory state of recent words encountered to
    incorporate word-order of short and long distances. This paper initially
    believes that models which maintain and understand word-order will perform
    better. 
    % TODO: Do something about the findings
    \item \textbf{Does tree structure improve the accuracy as opposed to
    sequential learning?} Vanilla LSTM as originally introduced, behaves in a
    sequential manner. One alternative to LSTM proposed was Tree LSTM, which can
    use multiple prior cells to infer the state of the current word, [enhancing
    the context available for the model]. The paper seeks to investigate whether
    adding these additional precursor cells for each prediction, will actually
    increase performance of the inference on the testing. The original belief is
    that this will certainly be the case, and as the results suggest, this
    [seems to be case].
    % TODO: Add cite for sequential introduction TODO: Add cite for tree
    % introduction TODO: Reflow bracketed text. TODO: Is this significant?
    \item \textbf{Performance on supervising sentiment per node?}
    % This paper is seeking to investigate if the Tree LSTM algorithm should be
    % trained on the supervision of the sentiment of individual words, or on the
    % sentiment of the sentences altogether. 
    This paper is seeking to investigate whether adding additional supervision
    on intermediate words or phrases will help the model capture the more
    context-related sentiment. The initial hypothesis of this paper argues
    that the supervision of individual nodes might decrease performance, due to
    less context to account for negating sentiment, for instance.
    % TODO: @Gongze, do you agree with the above initial hypothesis and
    % reasoning? ||| I want to add that the supervision is acted upon not only
    % word level, rather on a phrase level. And also the later result might
    % suggest a neutral effect of adding supervision, that is, no big gap
    % between these two cases. I suspect its because we dont train it with proper
    % learning rate or not until they converge. |||TODO: Incorporate later
    % results.
    \item \textbf{Will the performance vary much on different sentence length across models?}
    % A model's performance might depend on how many words the model needs to keep
    % in store. Decreasing the sentence length might show the difference in model
    % performance, and which model relatively performs better with longer
    % sentences. Initially, this paper argues that Tree LSTM might perform the
    % best, as it has more cells to infer relevant knowledge from than regular
    % LSTM. 
    A model may behave very differently to sentences of different length,
    because for longer sentences the sentiment semantics is more complex and
    sometimes contradict itself in different parts. So whether a model performs
    evenly on different length sentences and whether some model performs better
    on capturing sentiments contexts in longer time scale is a crucial problem.
    Initially, this paper argues that Tree LSTM might perform the best, as it
    process the context in a hand-crafted structural way, thus more efficient
    when dealing with long-scale semantics.
    % TODO: @Gongze, do you agree with the initial hypothesis? TODO: Add results
    % Might have inproper grammar and usage of word.
\end{enumerate}

% TODO: Add citation for Glove
In order to do so, the different models mentioned will be initialized with
300-dimensional GloVe\cite{pennington2014glove} vectors. The goal is to predict
the sentiment of [Stanford Sentiment Treebank data], with the different settings
corresponding to the [aforementioned research questions].
% SST doesn't have a paper to cite, weird.